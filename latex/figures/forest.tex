\begin{figure*}[t]
\centering
\resizebox{\textwidth}{!}{
\begin{forest}
for tree={
    grow=east,
    parent anchor=east,
    child anchor=west,
    draw,
    rounded corners,
    align=center,
    l sep+=10pt,
    s sep+=5pt,
    font=\sffamily,
    inner xsep=10pt,
    inner ysep=4pt,
    fill=blue!10,
    draw=blue!50,
    line width=1pt,
    edge path={
        \noexpand\path [draw=blue!50, thick, -] (!u.south east) |- (.child anchor)\forestoption{edge label};
    }
}
[Agent Architecture
    [Memory
        [Memory Sources
            [User-Agent Interaction
                [\textit{e.g. Xu et al. (2022), Jiang et al., RoleInteract (Chen et al., 2024a)}]
            ]
            [Agent-Agent Interaction
                [\textit{e.g. Maas et al. (2023), Generative Agent (Park et al., 2023)}]
            ]
        ]
        [Memory Usage
            [Retrieval-based
                [\textit{e.g. PLATO-LTM (Xu et al., 2022), Bae et al. (2022)}]
            ]
            [Compressive-based
                [\textit{e.g. COMEDY (Chen et al., 2024b), Wang et al. (2023d)}]
            ]
        ]
    ]
    [Planning
        [Planning Formulation
            [\textit{e.g. Dasgupta et al. (2023), Wang et al. (2023d), Inner Monologue (Huang et al., 2022a)}]
        ]
        [Planning Reflection
            [\textit{e.g. Generative Agent (Park et al., 2023), MORTISE (Tang et al., 2024), Wu et al. (2022)}]
        ]
    ]
    [Action
        [\textit{e.g. RoleLLM (Wang et al., 2023e), Humanoid Agents (Wang et al., 2023g), Shen et al. (2024a)}]
    ]
]
\end{forest}
}
\caption{The main content flow and categorization of Agent Architecture.}
\label{fig:agent-architecture}
\end{figure*}